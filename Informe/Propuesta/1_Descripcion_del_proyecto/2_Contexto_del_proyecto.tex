El plan estratégico de la Universidad de Talca, destaca el apoyo a todo tipo de actividades emprendidas por las distintas Organizaciones Estudiantiles reconocidas. Entre las actividades destacan: recepción de alumnos nuevos, celebración del día de la carrera, actividades culturales y deportivas, actividades sociales, entre otras. \cite{5}

Para la realización de estas actividades se debe seguir un procedimiento, el cual está estipulado en la RU2083 creada el 12 de diciembre del 2017, la cual dice:

\begin{tasks}[counter-format = {tsk[A].}]
	\task \textbf{Caso Federación y Organizaciones Estudiantiles:}

	El Presidente, Tesorero o Secretario Financiero, es quien deberá solicitar por escrito los Fondos aprobados para cada actividad al Director de DAAE, de la Vicerrectoría de Desarrollo Estudiantil, al menos 20 días previo al inicio de la actividad.

	El Director DAAE da visado a la solicitud de fondo para luego ser dirigida al Vicerrector de Desarrollo Estudiantil.

	El Vicerrector de Desarrollo Estudiantil luego de aprobar la solicitud de fondo, deberá emitir una Resolución Exenta, original con cuatro copias y la enviará a la Contraloría de la Universidad de Talca para el respectivo control de legalidad. Una vez totalmente tramitado el acto administrativo, será distribuido de la forma siguiente: Original para el archivo de la Secretaría General, la 1ra copia  para el Departamento de Tesorería y Presupuesto, la 2da, copa para el archivo de la Vicerrectoría Estudiantil, la 3ra, para la O.E. interesada y la 4ta para la unidad de Gobierno Transparente.

	\task \textbf{Caso de Centros de Alumnos}

	Se deberá efectuar por escrito una solicitud de fondos al Director de Escuela respectivo, quién elevará la solicitud al Decano de la Facultad correspondiente o al Vicerrector de Desarrollo Estudiantil para el caso de las Escuelas no adscritas a una Facultad.

	El Decano o Vicerrector de Desarrollo Estudiantil procederá a emitir una Resolución Exenta que transferirá los fondos para la actividad aprobada. Esta Resolución será emitida en original y cuatro copias, debiendo remitirse a la controlaría universitaria para el respectivo control de legalidad.

	Una vez aprobada por la Contraloría Interna y totalmente tramitada será enviada al Decano para su distribución de a siguiente forma: Original para Decanato, la 1ra copia para el Departamento de Tesorería y Presupuesto, la 2da copia para el archivo de la Facultad, la 3ra copia es para la O.E. interesada y la 4ta a la Unidad de Gobierno Transparente.

\end{tasks}

En ambos casos, el Departamento de Presupuesto imputará al centro de responsabilidad de la actividad presupuestaria correspondiente.

El Departamento de Tesorería emitirá el pago a nombre del Presidente o Tesorero o Secretario de Finanzas de la O.E. y adjuntará la 2da copia de la Resolución al Comprobante de Egreso para la posterior rendición del Fondo.

Al representante de la O.E. se le asignarán las siguientes responsabilidades:

\begin{itemize}
	\item Supervisar o realizar el cobro del documento de pago.
	\item Mantener un registro detallado de los gastos establecidos en el presupuesto de la actividad aprobada.
	\item Ejecutar/Realizar los gastos autorizados para la actividad.
	\item Rendir el Fondo asignado para la actividad.
	\item Respaldar gastos con la documentación idónea, tal como establece la Resolución Universitaria N\grad 522/1992, punto N\grad 12. 
\end{itemize} 

