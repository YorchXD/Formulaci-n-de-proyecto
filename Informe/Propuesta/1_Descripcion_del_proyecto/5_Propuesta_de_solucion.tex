Para resolver el problema planteado anteriormente es que se propone realizar un sistema web que permita a las distintas O.E. la creación de rendiciones, las cuales cumplan con el formato estipulado por la RU N\grad 2083.

Esta debe señalar los posibles errores por los cuales una rendición pueda ser rechazada, como por ejemplo, montos superiores a lo solicitado, boletas fuera del plazo en que se realiza la actividad, entre otros.

Además los usuarios pueden tener un respaldo de las resoluciones realizadas.

También se considera tener un instructivo de cómo realizar una rendición según los estándares que estipula la Universidad.   

Por otra parte, se requiere que al ingresar todas las boletas, buscar el óptimo monto de la suma de estas que sean igual o lo más cercana al monto solicitado.

En caso de que una rendición no sea completada, esta se guarde pero con estado pendiente.

La aplicación debe verificar que la rendición se realice dentro del periodo estipulado por la RU N\grad 2083 del año 2017 la cual dice que son 20 días contado desde el último acto administrativo de la transferencia de recursos.