%\emph{(En esta subsección se deben describir y justificar los metodos de evaluación/validación que se aplicarán a lo largo del desarrollo del proyecto.)}

%\textbf{Por ejemplo:}
%\begin{itemize}
	%\item Metodología de evaluación basado en experimentos
	%\item Metodología de evaluación basado en casos de estudio
	%\item Metodología de evaluación basado action research
%\end{itemize}


La metodología de evaluación que se utilizará es la metodología con experimentos controlados (prueba que se realiza bajo condiciones totalmente constante, y en algunos caso se puede variar algún factor). La idea principal es realizar un muestreo de usuarios conocidos quieres utilizarán el software y realizaran pruebas de usabilidad simple, además del cliente. 

Los usuarios que realicen las pruebas de usabilidad no necesariamente serán los mismo.

Se escoge esta metodología debido a que hace posible la comprensión e identificación de las variables que están siendo utilizadas en la construcción del software. Según el artículo \emph{Albert Einstein and Empirical Software Engineering} \cite{8}, experimentar con la construcción de software permitirá ``Aumetar la compresion de lo que hace al software bueno y cómo hacer un software bien''\footnote{Gain more understanding of what makes software ``good'' and how to make software well.}


