\documentclass[11pt,letterpaper]{article}
\usepackage{pslatex}
\usepackage[spanish]{babel}
\usepackage[utf8]{inputenc} % Caracteres con acentos.
\usepackage{latexsym}
\usepackage{amssymb} 
\usepackage{amsmath}
\usepackage{epsfig}
\usepackage{url}
\usepackage{subfiles}
\usepackage{tasks}
\newcommand{\grad}{$^{\circ}$}

\begin{document}

\pagestyle{empty}

\title{
Sistema de gestión de rendiciones para Federación y CCAA de la Facultad de Ingeniería de la Universidad de Talca\\
(Propuesta de proyecto final de carrera)}

\author{
Yorch Wilian Sepúlveda Manríquez (Estudiante)\\
Yamisleydi Salgueiro Sicilia (Profesor guía)\\
Carrera de Ingeniería Civil en Computación\\ 
Universidad de Talca}

\date{\today}

\maketitle


\section{Descripción de la propuesta}

\subsection{Conceptos básicos del proyecto} 
\subfile{1_Descripcion_del_proyecto/1_Conceptos_basicos}

\subsection{Contexto del proyecto} 
\subfile{1_Descripcion_del_proyecto/2_Contexto_del_proyecto}

\subsection{Definición del problema} 
\subfile{1_Descripcion_del_proyecto/3_Definicion_del_problema}

\subsection{Trabajo relacionado} 
\subfile{1_Descripcion_del_proyecto/4_Trabajo_relacionado}

\subsection{Propuesta de solución}
\subfile{1_Descripcion_del_proyecto/5_Propuesta_de_solucion}

%\section{Hipótesis}
%\emph{(En esta sección se deben incluir una lista de afirmaciones o suposiciones las cuales se esperar responder con el desarrollo del proyecto. La longitud máxima de esta sección es de 1/2 página.)}
%\begin{itemize}
%\item El uso de ... puede facilitar ....
%\item El problema de .... puede estudiarse como ... 
%\item Las técnicas usadas en ... pueden ser aplicables para resolver el problema de ...
%\end{itemize}



\section{Objetivos}
%\emph{(En esta sección se deben especificar el objetivo general y los objetivos específicos del proyecto. Los objetivos deben reflejar lo que se espera lograr con el proyecto, evitando incluir características específicas de la solución. La longitud máxima de esta sección es de 1 página.)}

\subfile{2_Objetivos/1_Objetivo_general}
\subfile{2_Objetivos/2_Objetivo_especifico}


\section{Alcances}
\subfile{3_Alcances/3_Alcances}



\section{Metodología}
%\emph{(En esta sección se deben describir y justificar los métodos que se usarán en el desarrollo del proyecto de titulación. Los métodos obligatorios que debe incluir esta seccion es: Metodología de Desarrollo/Investigación y Metodología de Evaluación.)}

\subsection{Metodología de desarrollo/investigación}
\subfile{4_Metodologia/1_Metodologia_de_desarrollo}


\subsection{Metodología de evaluación del proyecto}
\subfile{4_Metodologia/2_Metodologia_de_evaluacion_del_proyecto}

%\part{%\paragraph{Objetivo 1:} ``Comparación de algoritmos para ..."
%\begin{itemize}
%\item Estudiar ...
%\item Seleccionar ...
%\item Comparar ...
%\end{itemize}
%
%\paragraph{Objetivo 2:} ``Especificación de requisitos del software"
%\begin{itemize}
%\item Analizar ...
%\item Clasificar ...
%\item Especificar ...
%\end{itemize}
%}

%\section{Plan de trabajo}
%\subfile{5_Plan_de_trabajo/5_Plan_de_trabajo}


\bibliographystyle{plain}
\bibliography{bibliografia}



\end{document}