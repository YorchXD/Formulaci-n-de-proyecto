\documentclass[11pt,letterpaper]{article}
\usepackage{pslatex}
\usepackage[spanish]{babel}
\usepackage[utf8]{inputenc} % Caracteres con acentos.
\usepackage{latexsym}
\usepackage{amssymb} 
\usepackage{amsmath}
\usepackage{epsfig}
\usepackage{url}
\usepackage{subfiles}
\usepackage{tasks}
\newcommand{\grad}{$^{\circ}$}

\begin{document}

\pagestyle{empty}

\title{
    Gestión de fondos para actividades estudiantiles conducidas por CCAA o la Fedeut Curicó en acuerdo a la RU N\grad 2083 del año 2017\\
    (Propuesta de proyecto final de carrera)}

\author{
Yorch Wilian Sepúlveda Manríquez (Estudiante)\\
Rodrigo Paredes Moraleda (Profesor guía)\\
Carrera de Ingeniería Civil en Computación\\ 
Universidad de Talca}

\date{\today}

\maketitle


\section{Descripción de la propuesta}

Las Organizaciones Estudiantiles de la Universidad de Talca son las encargadas de gestionar muchas actividades para la comunidad estudiantil tales como la bienvenida de alumnos nuevos, aniversario de carrera, actividades culturales y deportivas, entre otras, y para ello se debe realizar un conjunto de procedimientos tales como la aprobación de solicitud, envío de la Resolución Universitaria, Rendiciones de Cuentas, entre otros. Pero la finalidad de esta propuesta está en el proceso de Rendiciones que detallan los gastos incurridos de las actividades realizadas por las Organizaciones Estudiantiles, en donde se centra el gran problema y es el que se pretende solucionar. Por lo tanto, en las siguientes páginas se muestra la propuesta de solución en donde se puede observar los Conceptos Básicos del Proyecto, el Contexto de trabajo de las Organizaciones Estudiantiles, la Descripción del Problema, Trabajos Relacionados, entre otros.



\subsection{Conceptos básicos del proyecto} 
\subfile{1_Descripcion_del_proyecto/1_Conceptos_basicos}

\subsection{Contexto del proyecto} 
\subfile{1_Descripcion_del_proyecto/2_Contexto_del_proyecto}

\subsection{Definición del problema} 
\subfile{1_Descripcion_del_proyecto/3_Definicion_del_problema}

\subsection{Trabajo relacionado} 
\subfile{1_Descripcion_del_proyecto/4_Trabajo_relacionado}

\subsection{Propuesta de solución}
\subfile{1_Descripcion_del_proyecto/5_Propuesta_de_solucion}

%\section{Hipótesis}
%\emph{(En esta sección se deben incluir una lista de afirmaciones o suposiciones las cuales se esperar responder con el desarrollo del proyecto. La longitud máxima de esta sección es de 1/2 página.)}
%\begin{itemize}
%\item El uso de ... puede facilitar ....
%\item El problema de .... puede estudiarse como ... 
%\item Las técnicas usadas en ... pueden ser aplicables para resolver el problema de ...
%\end{itemize}



\section{Objetivos}
%\emph{(En esta sección se deben especificar el objetivo general y los objetivos específicos del proyecto. Los objetivos deben reflejar lo que se espera lograr con el proyecto, evitando incluir características específicas de la solución. La longitud máxima de esta sección es de 1 página.)}

\subfile{2_Objetivos/1_Objetivo_general}
\subfile{2_Objetivos/2_Objetivo_especifico}


\section{Alcances}
\subfile{3_Alcances/3_Alcances}



\section{Metodología}
%\emph{(En esta sección se deben describir y justificar los métodos que se usarán en el desarrollo del proyecto de titulación. Los métodos obligatorios que debe incluir esta seccion es: Metodología de Desarrollo/Investigación y Metodología de Evaluación.)}

\subsection{Metodología de desarrollo/investigación}
\subfile{4_Metodologia/1_Metodologia_de_desarrollo}


\subsection{Metodología de evaluación del proyecto}
\subfile{4_Metodologia/2_Metodologia_de_evaluacion_del_proyecto}

%\part{%\paragraph{Objetivo 1:} ``Comparación de algoritmos para ..."
%\begin{itemize}
%\item Estudiar ...
%\item Seleccionar ...
%\item Comparar ...
%\end{itemize}
%
%\paragraph{Objetivo 2:} ``Especificación de requisitos del software"
%\begin{itemize}
%\item Analizar ...
%\item Clasificar ...
%\item Especificar ...
%\end{itemize}
%}

%\section{Plan de trabajo}
%\subfile{5_Plan_de_trabajo/5_Plan_de_trabajo}


\bibliographystyle{plain}
\bibliography{bibliografia}



\end{document}