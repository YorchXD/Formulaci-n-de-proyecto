%\emph{(En esta sección se debe incluir una lista de puntos que definen los límites del trabajo. La longitud máxima de esta sección es de 1/2 página.)}
%\begin{itemize}
%\item En este trabajo se espera implementar un prototipo funcional de la idea a desarrollar, por lo tanto se propone desarrollar una interfaz de línea de comando simple en lugar de una interfaz de usuario gráfica. 
%\item En este trabajo no se crearán algoritmos para ... 
%\item Este trabajo se limita a ...
%\end{itemize}


Se realizará un sistema web de gestión de rendiciones para Federación y CCAA de la Facultad de Ingeniería de la Universidad de Talca, en el cual se podrán realizar las siguientes funciones:

\begin{itemize}
	\item Gestionar rendiciones.
	
	\item Búsqueda de rendiciones realizadas por filtrado tales como fecha o número de rendición.
	
	\item Visualización detallada de rendiciones
	
	\item Búsqueda de suma montos de gastos de la rendición, similar o igual al monto solicitud más no superior.
	
	\item Validación de fechas de boletas ingresadas dentro de la actividad realizada.
\end{itemize}


%\begin{itemize}
	%\item Los usuarios podrán crear una rendición en la cual contemple fecha del documento, nombre del proveedor, breve descripción del gasto, monto del gasto.

	%\item Los usuarios podrán crear ademas a personas que hayan participado en la actividad, y detallar los gastos que tuvo en esta.

	%\item Los usuarios podrán revisar rendiciones anteriores (si es que existen).

	%\item Una vez finalizada una rendición, no se podrá editar.

	%\item Los usuarios podrán buscar rendiciones creadas con anterioridad a través de filtros tales como, fecha o numero de rendición.

	%\item El sistema debe validar de que se cumplan con los montos solicitados y que las fechas de las boletas ingresadas estén dentro del periodo de la actividad realizada.

	%\item El sistema debe tener un algoritmo en el cual se busque la suma montos de gastos de la rendición, similar o igual al monto solicitud más no superior.
%\end{itemize}