%\emph{(En esta sección se debe incluir una lista de puntos que definen los límites del trabajo. La longitud máxima de esta sección es de 1/2 página.)}
%\begin{itemize}
%\item En este trabajo se espera implementar un prototipo funcional de la idea a desarrollar, por lo tanto se propone desarrollar una interfaz de línea de comando simple en lugar de una interfaz de usuario gráfica. 
%\item En este trabajo no se crearán algoritmos para ... 
%\item Este trabajo se limita a ...
%\end{itemize}


Se realiza un sistema web de gestión de rendiciones para Federación y CAA de la Facultad de Ingeniería de la Universidad de Talca, en el cual se podrán realizar las siguientes funciones:

\begin{itemize}
	\item Realizar solicitudes de fondos por rendir que detallen en que se utiliza el dinero, participantes de la actividad, periodo en la que se realiza y responsable.

	\item Gestionar rendiciones de fondos por rendir que detallen los gastos incurridos de una actividad tras ser aprobada su correspondiente solicitud.
	
	\item Búsqueda de rendiciones realizadas por filtrado tales como fecha o número de rendición.
	
	\item Visualización detallada de rendiciones.
	
	\item Búsqueda de suma de montos de gastos de la rendición, similar o igual al monto solicitud, mas no superior.
	
	\item Validación de fechas de boletas ingresadas dentro de la actividad realizada.
\end{itemize}

\begin{itemize}
	\item 	\begin{description}
			    \item[Nota:] Cabe destacar que el sistema opera bajo la RU N\grad 2083 del año 2017.
			\end{description}
\end{itemize}

