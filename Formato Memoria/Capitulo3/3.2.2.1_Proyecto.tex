Para comenzar el proyecto, lo primero que se realiza es identificar el problema para ver cuál es el enfoque en el que se debe solucionar. Luego se procede a capturar las Historias de Usuarios que representan las características del proyecto. Tras tener una noción general de lo que el proyecto se procede a categorizar las Historias de Usuario creando los sprints. Estos son los que dan vida al proyecto a medida que se van terminando los sprints y cumpliendo con los tiempos planificados y siguiendo las fases que se muestran el en la \textbf{Figura \ref{fig: Proceso_Proyecto}} (en la siguiente sección se habla con más detalle sobre esto). 

Para efectos de esta memoria, los sprints están planificado de la siguiente manera:

\begin{enumerate}
    \item Solicitud
    \item Solicitud y Resolución
    \item Búsqueda
    \item Rendición
    \item Usuarios
    \item Documentación y Panel de ayuda
    \item Detalles
\end{enumerate}

Por último, tras cumplir con todos los sprints mencionados anteriormente se realiza la entrega del producto y la puesta en marcha.