Como se explicó en el punto anterior, cada proyecto comprende de muchas ``Entregas'', y cada una de ellas se realiza a través de lo que se denomina ``Iteraciones'' como se logra apreciar en la \textbf{Figura \ref{fig: Proceso_Proyecto}}. Este ciclo ayuda a tener control sobre el proceso de desarrollo y a disminuir los riesgos. Por otra parte, permite adaptar los cambios de requisitos en caso de haberlo. 

En cuanto a la duración y estructura de cada iteración depende de las características del proyecto. Para efecto de esta memoria, cada iteración tiene un tiempo aproximado de tres semanas y está compuesta por fases (Requisitos, Diseño, Planificación, Construcción y Pruebas) las cuales se hablará en más detalle en el siguiente ítem.


