Tras detectar el problema, lo primero que se realizó fue estudiar cómo realizar el proceso de solicitud de fondo por rendir, el cual se encuentra en la RU N\grad 2083 del año 2017. También se observó cómo las distintas OE realizan este proceso y así detectar las principales falencias que existen. Con todo lo mencionado anteriormente se procede a realizar el diagrama de todo el proceso que conlleva una solicitud de fondo por rendir para las distintas OE a la cual está enfocada la aplicación. Esto con el propósito de conocer y conseguir la mayor cantidad de requisitos e información del sistema.

Luego de obtener los requisitos que debe cumplir el sistema se realizó el diseño de la aplicación el cual contempla la línea de navegación, la estructura general, la base de datos y el diseño de interfaz.

Naturalmente, antes de comenzar todo el desarrollo de la aplicación, se procedió a realizar una planificación general del proyecto a través de iteraciones. Estas fueron planificadas en secuencia según el proceso que debe tener un fondo por rendir y las funcionalidades que conlleva cada una de estas.

En la primera iteración se deben cumplir todas las fases que tienen las iteraciones según lo estipulado en la \textbf{Figura \ref{fig: Proceso_Proyecto}}, dado a que el proyecto está en su etapa inicial y después, dependiendo de las circunstancias en que se encuentre el proyecto es que se ve si se siguen todas las etapas de las iteraciones. 

Si bien existe una planificación inicial, esta se puede modificar (y de hecho así lo fue) a medida que avanza el proyecto. Es por ello por lo que, en la etapa de requisitos de las iteraciones, las historias de usuarios que tienen planificada realizar en esta, se deben ver en detalle y confirmar con el cliente. También se realizan entrevistas a las Secretarias de Escuelas, dado a que asisten a los respectivos Directores de Escuela en la recepción y envío de solicitudes de fondos por rendir junto con sus respectivas rendiciones.

Luego se procede a detallar el diseño según la etapa en que se encuentra enfocada la iteración. En efecto, a partir de la segunda iteración, puede omitirse la fase de diseño cuando no hay cambios en los requisitos. Pero si se efectúan o detectan cambios de requisitos en la iteración anterior, estos deben ser analizados en la fase de requisitos de la iteración actual y luego se debe verificar si afecta o no al diseño para ser implementados.

Tras tener claro lo que se debe realizar, se procede a la etapa de planificación en donde cada historia de usuario se detalla en tareas las cuales darán vida al proyecto. Después, estas tareas pasan a la etapa de construcción en donde comienza a tomar forma el proyecto y tras terminadas todas las funcionalidades planificadas en la iteración, se procede a realizar las pruebas para corroborar que cumpla con lo estipulado por el cliente y la RU mencionada con anterioridad. Si todo resulta bien, se procede a realizar la entrega. En caso contrario, queda como retraso y se debe terminar en la siguiente iteración junto con las nuevas tareas que se asignan en esta última.

Dado a que el autor de la aplicación cumple con la mayoría de los roles es que las tareas deben planificarse de tal manera que se deban cumplir en el tiempo de tres semanas por iteración. Además, las preguntas que se deben realizar en las reuniones diarias de Scrum (como se explicó en la \textbf{Sección \ref{sec:Reuniones})}, en un comienzo se cumplían todos los días en que se trabajaba en el desarrollo de la aplicación. Pero con el avance del proyecto bajan a una vez por semana. Este ajuste se realiza debido a que no se requiere dar información del avance a otras personas debido a que los roles (a excepción del Stakeholder) lo realiza el autor de esta memoria, el cual conoce el avance del proyecto y los problemas que existen.

En cuanto a lo referente con el cliente, se está en comunicación con él al inicio de cada iteración cuando se debe confirmar los requisitos y al final, cuando se realiza la entrega de la iteración para que realicen algunas pruebas controladas. Una vez que se termina la versión beta (versión con funcionalidades principales), se realizan las pruebas con los distintos usuarios.