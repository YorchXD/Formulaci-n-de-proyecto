El proyecto comprende de muchas ``Entregas'', y cada una de ellas se realiza a través de lo que se denomina ``Iteraciones'' o ``Sprints'', como se menciona en la \textbf{Sección \ref{sec:Proyecto}}. Es por ello que para el desarrollo del sistema se define lo siguiente:

\begin{itemize}
    \item Cada iteración tiene un tiempo aproximado de \textbf{tres} semanas.
    \item Las iteraciones están compuestas por fases de requisitos, diseño, planificación, construcción y pruebas (ver \textbf{Sección \ref{sec:Fases}}).
    \item Cada iteración tiene un objetivo a cumplir según la funcionalidad del sistema. Para este proyecto se han planificado \textbf{siete} sprints.
\end{itemize}

Para organizar cada sprint que corresponde al desarrollo del sistema, existe una etapa previa la cual busca capturar las historias de usuarios que ayudan a tener una visión de lo que es el proyecto, permitiendo realizar el diseño del sistema. Tras lo mencionado anteriormente, se procede al inicio de la etapa de construcción.

Los sprint se organizan de acuerdo con la funcionalidad de la aplicación quedando de la siguiente forma:

\begin{enumerate}[label=\textbf{\arabic*}.]
    \item \textbf{Solicitud:} iteración centrada en la confección e implementación del proceso de Solicitud de Fondos por Rendir. Este entregable debe solicitar al usuario todos los datos que requiere un evento tales como categorías, participantes (si es que es enfocado a personas en específico), los datos principales del evento (nombre, fecha, etc.) y responsable del evento, entre otros.
    \item \textbf{Solicitud y Resolución:} iteración centrada en finalizar el proceso de solicitud generando un PDF con la solicitud realizada y el posterior ingreso tanto de los datos principales como de la copia digitalizada de la RU que envía la Universidad a la OE interesada.
    \item \textbf{Búsqueda:} iteración encargada de generar el entregable de búsqueda de Solicitudes tanto pendientes como finalizadas a través de fecha de creación de Solicitud o por nombre del evento.
    \item \textbf{Rendición:} iteración centrada en la confección de la rendición, la cual cuenta con el ingreso de los documentos (boletas y/o facturas), listado de la documentación según el orden en el que se ingresa para que el usuario las chequee y el documento de la Rendición en PDF.
    \item \textbf{Usuario:} iteración en donde se crea al administrador de la aplicación, el cual debe generar a nuevos usuarios y agregar nuevas categorías que se pueden ingresar en la solicitud para detallar en qué incurrirán los gastos.
    \item \textbf{Documentación y Panel de ayuda:} iteración encargada de realizar la ayuda que puede encontrar el usuario en la aplicación, además de la RU por la cual se rige la aplicación.
    \item \textbf{Detalles:} iteración encargada de afinar los detalles de la aplicación tales como, visualizar el estado de una solicitud de Fondo por Rendir, tener un respaldo de las rendiciones anteriores realizadas y editar una Rendición dentro de 20 días contados corridos después de haber finalizado el evento.
\end{enumerate}

