Como se mencionó en el ítem anterior, cada iteración cuenta con las siguientes fases:

\begin{itemize}
    \item Requisito
    \item Diseño
    \item Planificación
    \item Construcción
    \item Pruebas
\end{itemize}

Sin embargo, su presencia y tiempo de dedicación a cada una depende de cada iteración. Por ejemplo, en la primera iteración se incluyen todas las fases debido a que se realiza un prototipo del proyecto. Mientras que existen otras en que solo serán para corregir o modificar alguna funcionalidad. Todo lo anterior depende de las condiciones presentes en la iteración.

Se ha hablado de que existen fases, pero no se sabe que se realiza en cada una de ellas, es por ello por lo que a continuación se procede a contextualizar de que trata cada una:

\paragraph{Fase de requisitos: } en esta etapa es donde el Product Owner se junta con el cliente para obtener información sobre las funcionalidades que tiene el proyecto (etapa comúnmente denominada ``Captura de Requisitos’’). Para esta fase se recomienda hacer pautas para mejorar el trabajo con el cliente.

\paragraph{Fase de diseño: } tras la obtención de requisitos, se procede a realizar el diseño del sistema, la cual incluye la base de datos, la interfaz y la estructura. Cabe destacar que cada diseño debe ser simple de comprender.

\paragraph{Fase de planificación: } Luego de tener claro de lo que se quiere al realizar la captura de requisitos y el diseño, se procede a describir las actividades a realizar. Para cada actividad se debe definir su duración, dependencia, la fecha de inicio y de fin. Y por último asignar estas tareas a un responsable de algún equipo de trabajo. Pero para efectos de esta memoria y como se ha mencionado con anterioridad, todas estas tareas recaen solo una persona que es el autor del proyecto.

\paragraph{Fase de construcción: } ya asignadas las tareas a los responsables, se procede a codificar e implementar las funcionalidades que logran cumplir con los requisitos planteados en la planificación. La finalidad de esta etapa es desarrollar un incremento como se ha explicado en la Sección \ref{sec:Artefactos} a través de iteraciones sucesivas.

\paragraph{Fase de pruebas: } En esta fase se evalúa la funcionalidad del incremento. Para ello se hacen pruebas de integración y validación junto con el cliente, quien realiza las pruebas de aceptación.
