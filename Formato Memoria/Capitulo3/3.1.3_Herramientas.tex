Para el desarrollo de cada Sprint, se utiliza una aplicación gratuita y Open Source llamada Taiga, la cual es una plataforma de gestión de proyectos para diseñadores ágiles.

Para comenzar a utilizar esta plataforma se deben registrar las historias de usuarios con su respectiva prioridad (etapa conocida como Product Backlog). Una vez realizado lo anterior, se proceden a crear los Sprint y por último se añaden las historias de usuarios a cada uno de estos tal como se muestra en la \textbf{Figura~\ref{fig: SprintBacklog}} y en caso de ser necesario, una historia del Prooduct Backlog puede ser subdividad para lograr un mejor enfoque y claridad en el desarrollo.

\begin{figure}[h!]
    \includegraphics[width=\textwidth]{Imagenes/SprintBacklog.png}
    \caption{\label{fig: SprintBacklog} Sprint Backlog.}
\end{figure}

Dentro de cada Sprint hay un Kanban, el cual ayuda a organizar el desarrollo de cada historia de usuario realizada en el Sprint Backlog, en donde divide el trabajo en pequeñas tarjetas o tickets.

Dado lo anterior es que se divide el proceso de trabajo en cinco columnas como se muestra en la \textbf{Figura~\ref{fig: kanbanSprint}}, en donde se encuentra:

\begin{itemize}
    \item   \begin{description}
                \item[Nueva:] Se ingresan pequeñas tarjetas o ticket de trabajo que ayudan a realizar el cumplimiento de la historia de usuario 
            \end{description}

    \item   \begin{description}
                \item[En curso:] sección en donde se encuentran las tareas que se estan realizando
            \end{description}

    \item   \begin{description}
                \item[Lista para Testear:] columna en la que se encuentran las tareas terminadas y en proceso de aceptación
            \end{description}

    \item   \begin{description}
                \item[Cerrada:] área en donde se encuentran las tareas que han sido finalizada y aceptadas.
            \end{description}
    
    \item   \begin{description}
                \item[Necesita Información:] sección en donde se encuentran las tareas que no ha logrado culminarse debido a que falta información para su realización.
            \end{description}
\end{itemize}

\begin{figure}[h!]
    \includegraphics[width=\textwidth]{Imagenes/Kanban.png}
    \caption{\label{fig: kanbanSprint} Proceso de trabajo de un Sprint.}
\end{figure}

Para el diseño de base de datos se usa una aplicación llamada draw.io para diseñar el driagrama Entidad/Relación y luego de diseñar el diagrama E/R se procede crear el modelo relacional, es por que aparece StarUML que permite la creacion de estas tabas de forma gráfica y exportar todo esto a código, el cual es importado en MySQL que es donde está alojada la base de datos.

StarUML no solo se utilizó para diseñar la base de datos, sino que también para la creación de la arquitectura lógica y el diagrama de clases. Este diseño se utilizó en un framework (ver Sección \ref{sec:Tecnologias}) el cual ayudó a que se desarrollará la aplicación.  Por otra parte, se utiliza primeramente el IDE Visual Studio para la creación del proyecto, pero a lo largo del desarrollo de este último se cambió de IDE a VSCode ya que había que utilizar una librería para la creación de archivos PDF y que Visual Studio no podía utilizar debido a que entraba en conflicto.

Por último, se utilizó Bizzagi Modeler para la creación de diagramas de flujo que permiten comprender el proceso de solicitud de fondos por rendir tanto para Federación como Centros del Alumnos.