%% inicio, la clase del documento es iccmemoria.cls
\documentclass{iccmemoria}
\setcounter{secnumdepth}{3} %%Cantidad maxima de subsecciones enumeradas
\newcommand{\grad}{$^{\circ}$}
%\usepackage[utf8]{inputenc} % Caracteres con acentos.
%%\usepackage{enumitem}
\usepackage[inline]{enumitem}
%% datos generales y para la tapa
\titulo{Gestión de fondos para actividades estudiantiles conducidas por CCAA o la Fedeut Curicó en acuerdo a la RU N\grad 2083 de 2017}
\author{Yorch Sepúlveda}
\supervisor{Rodrigo Paredes Moraleda}
\informantes
	{Profesor Informante 1}
	{Profesor Informante 2}
\adicional{(sólo por si se necesita agregar algún otro profesor)}
\director{Profesor del ramo Memoria de Título}
\date{Enero, 2019}

%% inicio de documento
\begin{document}

%% crea la tapa
\maketitle

%% dedicatoria
\begin{dedicatory}
Dedicado a ...
\end{dedicatory}

%% agradecimientos
\subfile{Agradecimientos}

%% indices
\tableofcontents
\listoffigures
\listoftables

%% resumen
\subfile{Resumen}

%% abstract

%% contenido del primer capítulo
\chapter{Introducción}
    \label{sec:Intro}
	\subfile{Capitulo1/1_Introduccion}

	\section{Contexto del proyecto}
	\label{sec:Contexto}
	\subfile{Capitulo1/1.1_Contexto}

	\section{Definición del problema}
	\label{sec:Problema}
	\subfile{Capitulo1/1.2_Definicion_Problema}

	\section{Presentación de la solución}
	\label{sec:Solucion}
	\subfile{Capitulo1/1.3_Presentacion_Solucion}

	\section{Objetivos}
	\label{sec:Objetivos}
	\subfile{Capitulo1/1.4_Objetivos}

	\section{Alcances}
	\subfile{Capitulo1/1.5_Alcances}

	\section{Limitaciones}
	\subfile{Capitulo1/1.6_Limitaciones}

	\section{Proyectos similares}
	\subfile{Capitulo1/1.7_Proyectos_Similares}

	\section{Descripción de contenidos}
	\subfile{Capitulo1/1.8_Descripcion_Contenidos}

%% contenido del segundo capítulo
\chapter{Antecedentes}
\subfile{Capitulo2/2_Introduccion}

	\section{Conceptos}
	\label{sec:Conceptos}
	\subfile{Capitulo2/2.1.1_Conceptos}

	\section{Tecnologías utilizadas}
	\label{sec:Tecnologias}
		\subfile{Capitulo2/2.2_Tecnologias_Utilizadas}
		\subsection{Frameworks}

		\subfile{Capitulo2/2.2.1_Framework}
		\subsection{Bibliotecas}

		\subfile{Capitulo2/2.2.2_Bibliotecas}
		
	\section{Metodología de desarrollo de software}
		\subfile{Capitulo2/2.3_Metodologia_desarrollo}

	\section{Metodología ágil}
	\subfile{Capitulo2/2.4_Metodologia_Agil}

		\subsection{Scrum}
		\subfile{Capitulo2/2.4.1_Scrum}

		\subsection{Roles}
		\label{sec:Roles}
		\subfile{Capitulo2/2.4.2_Roles}

		\subsection{Aptitudes requeridas}
		\subfile{Capitulo2/2.4.3_Aptitudes}

		\subsection{Artefactos}
		\label{sec:Artefactos}
		\subfile{Capitulo2/2.4.4_Artefactos}
		
		\subsection{Reuniones}
		\label{sec:Reuniones}
		\subfile{Capitulo2/2.4.5_Reuniones}
		
		\subsection{Ciclo de vida}
		\subfile{Capitulo2/2.4.6_Ciclo_de_vida}

		\subsubsection{El proyecto}
		\label{sec:Proyecto}
		\subfile{Capitulo2/2.4.7_Proyecto}

		\subsubsection{Las fases}
		\label{sec:Fases}
		\subfile{Capitulo2/2.4.8_Fases}

		\subsection{Documentación}
		\label{sec:Documentacion}
		\subfile{Capitulo2/2.4.9_Documentacion}

	\section{Validación de la aplicación}
		\subfile{Capitulo2/2.5_Pruebas}

	\section{Herramientas de planificación}
		\subfile{Capitulo2/2.6_HerramientasPlanificacion}
	
%% contenido del tercer capítulo
\chapter{Metodología utilizada}
\subfile{Capitulo3/3_Introduccion}

	\section{Estudio del entorno}
	\subfile{Capitulo3/3.1_Entorno}

		\subsection{Las personas}
		\subfile{Capitulo3/3.1.1_Personas}

		\subsection{La aplicación}
		\subfile{Capitulo3/3.1.2_Aplicacion}

		\subsection{Las herramientas}
		\subfile{Capitulo3/3.1.3_Herramientas}
	
	\section{La metodología}
	\subfile{Capitulo3/3.2_Metodologia}
		\subsection{Roles}
		\subfile{Capitulo3/3.2.1_Roles}

		\subsection{El proceso}
		\subfile{Capitulo3/3.2.2_Proceso}

			\subsubsection{El proyecto}
			\subfile{Capitulo3/3.2.2.1_Proyecto}

			\subsubsection{Iteraciones del proyecto}
			\label{sec:Iteracion}
			\subfile{Capitulo3/3.2.2.2_Iteracion}

		\subsubsection{Reuniones}
		\subfile{Capitulo3/3.2.3_Reuniones}

		\subsection{Cómo se trabajó}
		\subfile{Capitulo3/3.2.4_Trabajo}
	
%% contenido del Cuarto capítulo
\chapter{Características del sistema}
\label{sec:Caracteristica_Sistema}
\subfile{Capitulo4/4_Introduccion.tex}

	\section{Aspectos generales}
	\subfile{Capitulo4/4.1_AspectosGenerales}

	\section{Modelos de contexto de solicitudes}
	\subfile{Capitulo4/4.2_Modelos_Contexto_Solicitud.tex}

	\section{Características generales del software}
	\subfile{Capitulo4/4.3_CaracteristicasGenerales}

	\section{Planificación de las iteraciones}
	\subfile{Capitulo4/4.4_Planificacion}

	

%% contenido del quinto capítulo
\chapter{Diseño de la aplicación}
\subfile{Capitulo5/5_Introduccion}

	\section{Mapa de navegación}
	\subfile{Capitulo5/5.1_Mapa_navegacion}

	\section{Interfaces de usuario}
	\subfile{Capitulo5/5.2_Interfaces}

	\section{Arquitectura de la aplicación}
	\subfile{Capitulo5/5.3_Arquitectura_Aplicacion}

		\subsection{Arquitectura física}
		\subfile{Capitulo5/5.3.1_Arquitectura_fisica}

		\subsection{Arquitectura lógica}
		\subfile{Capitulo5/5.3.2_Arquitectura_logica}

	\section{Diagrama de clases}

	\section{Modelo de datos}

	

	

	

%% contenido del quinto capítulo
\chapter{Construcción y validación}

	\section{Aplicación obtenida}

	\section{Pruebas}

	\section{Resultados}





	












%\subsection{La primera subsección de la primera sección del capítulo 1}
%\subfile{1_Descripcion_del_proyecto/1_Conceptos_basicos}
%Aquí va el texto de la primera subsección de la primera sección del capítulo 1... 

%\subsection{La segunda subsección de la primera sección del capítulo 1}
%\subfile{1_Descripcion_del_proyecto/1_Conceptos_basicos}
%Aquí va el texto de la segunda subsección de la primera sección del capítulo 1...

%\section{La segunda sección del capítulo 1}
%\subfile{1_Descripcion_del_proyecto/1_Conceptos_basicos}
%Aquí va el texto de la segunda sección del capítulo 1...


%% contenido del segundo capítulo
%\chapter{Segundo Capítulo}
%\subfile{1_Descripcion_del_proyecto/1_Conceptos_basicos}
%Sólo para probar algunas cosas como las referencias.
%La primera cita es a Lamport~\cite{lamport79}.
%La segunda cita es para Lamport nuevamente~\cite{lamport78}.
%La última cita es para Keleher \emph{et al.}~\cite{keleher92}.


%% contenido del tercer capítulo
%\chapter{Tercer Capítulo}
%\subfile{1_Descripcion_del_proyecto/1_Conceptos_basicos}
%Sólo para incluir figuras y tablas.
%\begin{figure}[h]
%  \vspace*{1cm}
%  \includegraphics[bb=0 0 640 480, width=.5\linewidth]{latexlogo.png}
 % \vspace*{1cm}
%  \caption{La primera figura de la memoria}
%\end{figure}
%\begin{table}[h]
%  \vspace*{1cm}
%  (aqui debiera ir la tabla)
%  \vspace*{1cm}
%  \caption{La primera tabla de la memoria}
%\end{table}


%% ambiente glosario
\subfile{Glosario}



%% genera las referencias
\bibliography{refs}


%% comienzo de la parte de anexos
\appendixpart

%% contenido del primer anexo
\appendix{El Primer Anexo}
Aquí va el texto del primer anexo...

\section{La primera sección del primer anexo}
Aquí va el texto de la primera sección del primer anexo...

\section{La segunda sección del primer anexo}
Aquí va el texto de la segunda sección del primer anexo...

\subsection{La primera subsección de la segunda sección del primer anexo}


%% contenido del segundo anexo
\appendix{El segundo Anexo}
Aquí va el texto del segundo anexo...

\section{La primera sección del segundo anexo}
Aquí va el texto de la primera sección del segundo anexo...

%% fin
\end{document}

   

