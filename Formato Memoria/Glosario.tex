\begin{glosario}
	\item 	\begin{description}
		\item[Burocracia:] Administración ineficiente a causa del papeleo, la rigidez y las 
		formalidades superfluas~\cite{16}.
	\end{description}

	\item 	\begin{description}
		\item[Burocrático(ca):] adj. Perteneciente o relativo a la burocracia~\cite{15}.
	\end{description}

	\item 	\begin{description}
		\item[CAA:] Centro de alumnos. Organización estudiantil que representa a los alumnos de una carrera en particular~\cite{3}.
	\end{description}

	\item 	\begin{description}
		\item[DAAE:] Dirección de Apoyo a Actividades Estudiantiles. Unidad perteneciente a la Vicerrectoría de Desarrollo Estudiantil (VDE) que promueve el desarrollo integral de los estudiantes, mediante la entrega de herramientas complementarias a su formación académica, que les permitan adquirir las competencias necesarias para lograr un proyecto de vida personal y profesional exitoso, haciendo de la etapa universitaria una experiencia más enriquecedora~\cite{1}.
	\end{description}

	\item 	\begin{description}
		\item[Decano:] Autoridad superior de la Facultad y dirige todos los asuntos académicos, administrativos y financieros de ella~\cite{1}.
	\end{description}

	\item 	\begin{description}
		\item[Director de Escuela:] Académico que con dedicación preferente, está encargado de la gestión del plan de estudios de la(s) Carrera(s) a su cargo. Es designado por el Rector a proposición del Decano y depende jerárquicamente de éste ultimo~\cite{1}.	
	\end{description}

	\item 	\begin{description}
		\item[Engorroso(sa):] adj. Dificultoso, molesto~\cite{14}.
	\end{description}

	\item 	\begin{description}
		\item[Escuela:] Unidad básica de administración de uno o más programas docentes afines de pregrado. Esta a cargo de un(a) Director(a) de Escuela, quien depende jerárquicamente de un(a) Decano(a), o del Vicerrector(a) de Pregrado cuando se trate de carreras no adscritas a una Facultad. Excepcionalmente una Escuela puede depender del Rector(a)~\cite{1}.
	\end{description}

	\item 	\begin{description}
		\item[Facultad:] Unidad académica que, en conformidad con el Estatuto y las Ordenanzas de la Universidad, agrupa a un cuerpo de personas asociadas con el propósito de enseñar e investigar en áreas afines del conocimiento superior. Una Facultad está dirigida por un Decano~\cite{1}.
	\end{description}

	\item 	\begin{description}
		\item[Federación de Estudiantes:] Organización estudiantil que representa a los alumnos de un campus perteneciente a la Universidad~\cite{2}.
	\end{description}

	\item 	\begin{description}
		\item[Fondos:] Recursos  económicos  que  la  Universidad  de  Talca  dispone  para  la  realización  de diversas iniciativas y/o actividades, entre ellas, las estudiantiles~\cite{1}.
	\end{description}

	\item 	\begin{description}
			\item[Fondo por Rendir:] Sumas de dinero solicitadas y puestas a disposición de los jefes de unidades y proyectos con presupuesto asignado y disponible, que necesiten atender gastos de carácter especial, imprevistos o urgentes, sujetos todos ellos a rendición posterior~\cite{1}.
		\end{description}
	
	\item 	\begin{description}
			\item[Grupos Itermedios:] Grupos de estudiantes que se reúnen con un objetivo y/o interés en común, los cuales son reconocidos por la Universidad de Talca~\cite{18}.
		\end{description}

	\item 	\begin{description}
			\item[OE:] Organizaciones Estudiantiles. Estas son federación de estudiantes, centros de alumnos o grupos intermedios~\cite{2}.
		\end{description}

	\item 	\begin{description}
			\item[Resolución:] Puede ser un decreto, una decisión o un fallo que emite una determinada autoridad. Estas pueden establecer reglas, voluntades, etc.
		\end{description}
	
	\item 	\begin{description}
			\item[Resolución Exenta:] Comprenden aquellas Resoluciones en que no es necesario que sean visadas por la Contraloría General de la República~\cite{3}.
		\end{description}
	
	\item 	\begin{description}
			\item[RU:] Resolución Universitaria.
		\end{description}

	\item 	\begin{description}
			    \item[Visar:] Dicho de la autoridad competente: Dar validez a un pasaporte u otro documento para determinado uso~\cite{4}.
			\end{description}

	\item 	\begin{description}
			    \item[Principio de optimalidad:] aplicado en programación dinámica, consiste en que una secuencia óptima de decisiones que resuelve un problema, debe cumplir la propiedad de que cualquier subsecuencia de decisiones, que tenga el mismo estado final, debe ser también óptima respecto al subproblema correspondiente.~\cite{21}.
			\end{description}  

\end{glosario}
