\begin{resumen}

    %Marco general: 2 a 3 lineas contando el marco general, contexto o motivacion del problema.

    El problema que esta memoria resuelve surge cuando el autor se encontraba en calidad de Presidente del Centro de Alumnos de Ingeniería Civil en Computación de la Universidad de Talca.
    
    %Definicion o planteamiento del problema: 3 a 5 lineas en donde cuentan cual es el problema que efectivamente estas enfrentando.
    
    El principal problema que existía en ese tiempo era que, si se rechaza una Rendición que justifica los gastos realizados en un evento por una Organización Estudiantil(OE) en Tesorería al momento de su revisión, esta no envía una notificación con los errores detectados, sino que usualmente notifica el primer error provocando el retraso tanto del reembolso del dinero como de otras actividades planificadas por la OE. Por consecuencia, no se pueden realizar nuevas Solicitudes de Fondos por parte de la OE mientras no culmine la Rendición pendiente.

    %El inconveniente nace cuando una Rendición que justifica los gastos realizados en un evento por una Organización Estudiantil es rechazada por Tesorería al momento de su revisión. Esta entidad no envía una notificación con los errores detectados, sino que usualmente notifica el primer error, provocando el retraso tanto del reembolso del dinero como de otras actividades planificadas por la Organizacion Estudiantil ya que no se pueden realizar nuevas Solicitudes de Fondos mientras no culmine la Rendición pendiente.
    
    %Metodologia: 3 a 5 lineas en que hablas de la metodologia de trabajo.
    Tras lo expuesto anteriormente es que se realiza una plataforma web. Para su construcción se utiliza la metodología de desarrollo ágil \emph{Scrum}, debido a que permite la adaptación al cambio a medida que se desarrolla el sistema en conjunto con retroalimentaciones que brindan los usuarios.
    
    %Antecedentes o Marco teorico: 3 a 8 lineas de cosas previas o relacionadas que hay que saber antes de enfrentar el resto del resumen.

    %Existen tres tipos de Organizaciones Estudiantiles las cuales son: Federación de Estudiantes, Centros de Alumnos y Grupos Intermedios. Pero esta herramienta está enfocada para que asista en el proceso de Solicitud de Fondos por Rendir para las dos primeras Organizaciones.
    
    %Desarrollo: 3 a 5 lineas explicando como resolviste el problema.

    Para resolver el problema se realizan encuestas a los clientes con el fin de obtener las características del sistema y el modelo de contexto para mayor comprensión del actual sistema. Luego, estas características son categorizadas de acuerdo con la utilidad que preste (planificación de un \emph{Sprint}). Además, se realiza el diseño de la aplicación la cual comprende la confección del mapa de navegación, diseño de interfaz, la arquitectura y el modelo relacional del sistema.

    %Para resolver el problema se realizan encuestas a los clientes con el fin de obtener las características del sistema y el modelo de contexto para la mayor comprensión del actual sistema. Luego, estas características son categorizadas de acuerdo con la utilidad que preste (planificación de un \emph{Sprint}). Además, se realiza el diseño de la aplicación la cual comprende la confección del mapa de navegación, diseño de interfaz, la arquitectura y el modelo relacional del sistema. 
    
    %Resultados: 2 a 5 lineas donde cuentan los resultados mas importantes de su memoria. Si pueden ponerle numeros al asunto mejor, pues es mas objetivo.

    Luego de desarrollar la aplicación se realizan pruebas de caja negra para contrastar que se cumpla con los objetivos, y pruebas de usabilidad que se realizan a los clientes en donde el 75\% de ellos expresa que es una herramienta necesaria para asistir en el proceso de Solicitud de Fondo por Rendir. Por otra parte, el 63\% de los encuestados dicen que la aplicación ayudaría a reducir la causa de rechazo de una rendición.

    %Luego de desarrollar la aplicación y el uso de esta por parte de los clientes, el 75\% de ellos expresa que es una herramienta necesaria para asistir en el proceso de Solicitud de Fondo por Rendir. Por otra parte, el 63\% de los encuestados dicen que la aplicación ayudaría a reducir la causa de rechazo de una rendición.
    
    %Validacion: 2 a 3 lineas donde explican como verifican que los objetivos de vuestra memoria fueron cumplidos (o no). Esto puede ser hablando de la evaluacion experimental de lo que hicieron, evaluacion de los resultados de la memoria, algun mecanismo de validacion de la aplicacion, encuestas de usuario o la forma que sea apropiada para vuestra memoria en especifico. Ojo que si los objetivos no se cumplen, deben tener una buena explicacion de porque no se cumplieron. Nuevamente,si pueden valorar con alguna cifra la validez de los resultados mejor, pues es mas objetivo.
    
    %Estos resultados se obtienen a través de la evaluación de la aplicación realizadas por pruebas de Caja Negra y una encuesta de usabilidad que se realiza al usuario.
    
    %Discusion de resultados: 1 a 5 lineas en donde hagan una interpretacion reflexiva de los resultados, tambien se puede ver que estan mostrando que significan los resultados, etc.
    
    %Dado que un gran porcentaje de Representantes mencionan que la herramienta ayudaría a reducir la causa de rechazo de una rendición,se puede inferir que la aplicación podría ser oficial para todas las Organizaciones Estudiantiles de la Casa de Estudio.

    %Conclusiones: 2 a 5 lineas donde esbozan los elementos mas importantes de las conclusiones de vuestro trabajo.
    
    Para concluir, se obtuvo una aplicación la cual asiste en el proceso de Solicitud y Rendición de un Fondo la cual está bajo los estándares impuestos por la Casa de Estudio. Como consecuencia de lo anterior, permite reducir los errores que podrían rechazar las Rendiciones.

    %Para concluir, se afirma que la aplicación tiene muchas opciones de mejora para obtener una herramienta la cual asista en el proceso de Solicitud y Rendición de un Fondo con mayor fiabilidad de resultados. Sin embargo, la aplicación contempla las funcionalidades principales y sigue los estándares impuestos por la Casa de Estudio. Como consecuencia de lo anterior, permite reducir los errores que podrían rechazar las Rendiciones.

    %Trabajo futuro: 2 a 5 lineas para delinear el trabajo futuro o de continuacion de vuestra memoria.

    %Las mejoras que se pueden realizar a futuro de la aplicación es realizar que las interfaces sean responsivas y que tengan mejor armonía de colores. Además, se necesita que la interfaz funcione para diferentes navegadores tales como Safari y Opera, entre otro. 

\end{resumen}