La última fase de las iteraciones es la etapa de prueba, es por ello que para cada una de ellas se realizan Pruebas de Caja Negra. Esto se realiza con la finalidad de corroborar que el sistema cumple con los requerimientos que el usuario solicitó. Principalmente, las pruebas más importantes son las del proceso de solicitud de fondo por rendir, la cual comienza con las fases de confección de la solicitud. Luego se procede a diseñar las pruebas para la aceptación de la solicitud y el ingreso de los datos de la RU enviada por la casa de estudio. Por último, se realizan las pruebas para confección de la rendición correspondiente. Para ello el sistema debe realizar el proceso de ingreso de los datos para boletas y/o facturas, a las cuales se le denominan Documentos. Además, se debe generar el documento de declaración de gastos del evento. Para más detalles de las Pruebas de Caja Negra vea \textbf{Anexo \ref{sec: Anexo_A}}.

Cabe destacar que para los Documentos se valida por un lado que la fecha de emisión del Documento esté comprendida dentro del periodo que comienza con la fecha de emisión de la RU respectiva y termina con la fecha de finalización del evento, y por el otro, que la fecha de declaración de Documento esté dentro del periodo de rendición de gastos (20 días corridos posterior al evento). Adicionalmente, se verifica que la suma del monto de los Documentos esté dentro del monto solicitado. 

En cuanto a las pruebas de usabilidad, se le entregan a las OE una serie de tareas para que interactúen con el sistema. Estas tareas son: 

\begin{enumerate}
    \item Confección de solicitud.
    \item Generar PDF de solicitud.
    \item Búsqueda de una Solicitud.
    \item Ingreso de datos de RU.
    \item Confección de Rendición.
    \item Generar PDF de Rendición.
    \item Generar PDF del listado de documentos.
\end{enumerate}

Para más detalle sobre la pruebas de Caja Negra realizadas, se puede consultar en el \textbf{Anexo \ref{sec: Anexo_A}}.

