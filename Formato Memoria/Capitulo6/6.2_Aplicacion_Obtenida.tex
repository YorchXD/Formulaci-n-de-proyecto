%En primer lugar, se expone el caso real de un estudiante al utilizar la herramienta, con el motivo de mostrar al lector su funcionamiento.

\paragraph{Autentificacion de un usuario en el sistema: } Solicitud de acceso utilizando credenciales de acceso los cuales son ingresados por el usuario. Si las credenciales son válidas, se redirecciona a las interfaces de la OE interesada o a la del Administrador.

\paragraph{Recuperación de acceso a una cuenta de una OE: } Solicitud de una nueva clave de acceso en caso de que el usuario la olvide ingresando el correo de la OE a la que pertenece. Una vez ingresada esta información se solicita la recuperación de cuenta al servidor, el cual envía un correo con la nueva clave de acceso al usuario.

\paragraph{Búsqueda de solicitudes: } En esta etapa de la aplicación se encuentran todas las solicitudes realizadas por la OE, en donde el usuario puede visualizar los resúmenes de la solicitudes y tener la opción de ver en detalle cada una de ellas. Además, se encuentra la opción de realizar búsquedas de solicitudes a través del nombre o la fecha del evento.

\paragraph{Solicitud: }Esta etapa consta de 4 fases las cuales ayudan a la confección de la solicitud a presentar a la Dirección correspondiente de cada OE a la que pertenece. Estas fases se detallan a continuación. 

    \subparagraph{Datos principales de la Solicitud: } Formulario que solicita el nombre, la fecha de inicio, fecha de término, el lugar y el monto del evento, el responsable a cargo de la solicitud y la cantidad de participante, la cual puede ser para un grupo de personas o masiva según sea el caso.

    \subparagraph{Categoría: } En esta etapa solicita todas las categoría de los gastos que incurren dentro del evento al cal se está realizando la solicitud.

    \subparagraph{Participante: } Formulario que solicita el nombre y Rut de los participantes de la actividad. Esta vista sólo se visualiza si en la etapa de \textbf{Datos principales de a solicitud} siempre y cuando no se indique que la actividad es masiva. 

    \subparagraph{Resumen: } Etapa que muestra los datos principales de la solicitud y a su vez da la opción de generar el PDF de a solicitud.

    \paragraph{Resolución: } En esta etapa se acepta o rechaza la solicitud. En caso de ser aceptada se procede a ingresar los datos principales de la RU enviada por la Casa de Estudios a la OE interesada los cuales son el número de la resolución y el año de creación. Además, se solicita que se adjunte una copia digitalizada y en formato PDF de la RU correspondiente a la aceptación de la aceptación de la Solicitud enviada por la OE.

\paragraph{Rendición: } Tras la aceptación de la Solicitud y el ingreso de los datos de la RU que la aprueba se procede a realizar la Rendición correspondiente al evento. Si bien los datos principales de la Rencicion tales como el nombre de la organización, el numero de la resolución y el responsable del evento, entre otros, se obtiene automaticamente de los datos registrados en el sistema, el monto total rendido tienen dependencia del registro de documentos (boletas y/o facturas).

    \subparagraph{Documento: } seccion en donde se ingresan los datos de una boleta y/o factura en donde se ingresan datos tales como el número del documento, la categoria en que incurrieron los gatos, la fecha de compra de los productos, el monto, una pequeña descripción de los gastos y proveedor, entre otros.



