La encuesta cuenta con tres secciones y fue realizada a los Presidentes(as) y Secretarios(as) de Finanzas de cada OE. Pero previo a esta, los representantes tuvieron la oportunidad de probar el sistema. Cabe destacar que estas pruebas están enfocadas en el proceso de una Solicitud de Fondo por Rendir.

A continuación, se da a conocer el resumen de los resultados de cada sección de la encuesta.

\paragraph{Sección de preguntas generales sobre el sistema}

El 75\% de los representantes de las OE expresan que es necesaria una herramienta que ayude a asistir en el proceso de realizar una Solicitud de Fondo por Rendir. Por otra parte, el 56\% de los encuestados indican que el sistema ayudaría a reducir el tiempo en que demanda en realizar una Rendición.

El 44\% de los encuestados indican que la mayoría de las personas pueden aprender a utilizar el sistema rápidamente mientras que el resto tienen ciertas demuestran ciertas dudas, pero aun así confía en el sistema.

Por último, el 63\% de los encuestados indican que la aplicación ayudaría a reducir la causa de rechazo de una Rendición, mientras que el resto demuestran no está completamente en acuerdo con esta afirmación, pero confían en el sistema entre un 60\% a un 80\%.

\paragraph{Sección de preguntas sobre las funcionalidades del sistema}

El 75\% de los representantes de las OE indican que las funcionalidades del sistema son suficientes para que la aplicación sea útil para ellos. Por otra parte, al momento de realizar las pruebas existieron fallas al momento de realizar una rendición, las cuales fueron resueltas con posterioridad. Sin embargo, el 69\% de los representantes indica que la aplicación logra realizar una Rendición fiable.

\paragraph{Sección de preguntas sobre usabilidad e interfaz de usuario}

El 63\% de los representantes indican que la aplicación es fácil de usar y que no requiere de instrucciones previas, mientras que el resto indican un pequeño desacuerdo. Por otra parte, el 62\% de los usuarios indican que la distribución de los elementos en la pantalla es adecuada, mientras que el resto no está completamente de acuerdo. Todo lo mencionado anteriormente se debe a que la aplicación, cuando se realizaron estas pruebas, estaba enfocada en la funcionalidad por sobre la interfaz, pero posteriormente se trabaja en las mejoras.

Por último, el 75\% de los representantes indican que el proceso completo de realizar una Solicitud de Fondo por Rendir es claro, mientras que el 25\% indican que la claridad de este proceso es de un 80\%.

Para más detalle sobre la encuesta realizada, se puede consultar en el \textbf{Anexo \ref{sec: Anexo_B}}.