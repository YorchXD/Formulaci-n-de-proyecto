\paragraph{Entorno de trabajo: } La estación de trabajo Utilizada para la construcción de la aplicación corresponde a un computador que cuenta con un procesador Intel CORE i7 de 2.80GHz, 12Gb de RAM y con conexión a Internet. El sistema operativo es Windows 10 Home, en el cual se realizan las pruebas.

En cuanto a los navegadores en el cual se realizaron las pruebas son: Mozilla Firefox Quantum 67.0.1 (64-bit) e Internet Explorer. Pero el navegador recomendado es Mozilla Firefox.

\paragraph{Entorno y lenguajes de programación: } Dado a que SimRend es una aplicación Web, se utiliza el framework ASP.NET Core y junto con ello se utilizan los lenguajes C\#, JavaScript, HTML y CSS (ver \textbf{Sección \ref{sec:Tecnologias}}). En primera instancia se utilizó el IDE Visual Studio, pero debido a que hubo un conflicto con el uso de la libreria DinkToPDF se migró a VSCode. La elección de estos IDE, se debe a la experiencia que a tenido el autor al realizar proyectos en su periodo académico dentro de la casa de estudio y porque no se exige algun IDE en particular.

La estructura de SimRend está hecha en HTML y el diseño es mejorado utilizando hojas de estilos (CSS), junto con la ayuda de JavaScript que permite controlar algunos eventos en el sistema. En cuanto a la capa de negocio y el modelo de datos se utiliza C\#. Todo lo anterior se une gracias a la arquitectura lógica del proyecto la cual es MVC, que es detallada en la \textbf{Sección \ref{sec:Arquitectura_Logica}}.

\paragraph{Motor y aplicación para base de datos: } Es un motor de base de datos relacional, la cual posee una gran estabilidad y confiabilidad además de fiabilidad e integridad de los datos. Para ello, la herramienta que se utiliza es StarUML y para la administración MySQL.


%En primer lugar, se expone el caso real de un estudiante al utilizar la herramienta, con el motivo de mostrar al lector su funcionamiento.

\begin{itemize}
    \item Credenciales de acceso
    \item Recuperación de contraseña
    \item Solicitud
    \begin{itemize}
        \item Datos principales del evento y responsable
        \item Categorias
        \item Participantes
    \end{itemize}
    \item Resolucion
    \item Rendición
    \begin{itemize}
        \item Documentos
    \end{itemize}
    
\end{itemize}



