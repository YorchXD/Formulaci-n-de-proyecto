A través de este documento se dan a conocer el proceso que conlleva el desarrollo de este \textit{Proyecto de Titulación}. En un comienzo se detalla el proceso que se debe realizar en una Solicitud de Fondo por Rendir junto con los conceptos principales que ayudan a la comprensión para el desarrollo de esta memoria. Luego se procede a describir la problemática que busca resolver este proyecto: dado lo engorros que es la confección y la aceptación de una Rendición ya que los representantes de las OE no siguen el proceso estipulado para la aceptación y su posterior devolución de dinero que cubren los gastos  incurridos en un evento es que se desarrolla un sistema el cual \textbf{Confecciona solicitudes de fondos por rendir para actividades estudiantiles y las rendiciones de gastos respectivas, de acuerdo con el marco definido por la RU N\grad 2083 del año 2017.}

Esta memoria detalla todo el proceso que conlleva el desarrollo de la aplicación el cual comienza con la captura de requisitos y la planificación general del proyecto la cual se modificó dado a que surgía problemáticas y/o nuevos requisitos. Luego, tras la planificación se comienza con el diseño de la aplicación y su posterior desarrollo que se realizaban a través de incrementos. Por último, se realizaban las pruebas de los incrementos para verificar si cumplía con el objetivo de cada iteración.

Por último, se realizan pruebas de Caja Negra para medir los objetivos planteados en este documento. Además, se realizaron pruebas de Usabilidad que debió ser contestado por los representantes de las distintas OE.

Tras realizar las diferentes pruebas a los usuarios de la aplicación, estos indican que la herramienta logra asistir en el proceso de una Solicitud de Fondo por Rendir reduciendo los posibles errores tales como el ingreso boletas y/o facturas fuera del periodo del evento, gastos incurridos sin categorización y monto excesivo en comparación al solicitado, entre otros. Esto no implica que el usuario pueda cometer errores los cuales producen el rechazo de la Rendición, pero es un factor que no puede ser controlado por la aplicación debido a que es un factor humano. 

Para finalizar, este proyecto entrega un sistema funcional el cual puede ser utilizado por los distintos CCAA y Federación de la Universidad de Talca, Campus Los Niches, Curicó. Pero además existe trabajo futuro que es posible realizar con este Software los cuales son:

\begin{itemize}
    \item Hacer que las interfaces del sistema sean responsivas.
    \item Implementar que el sistema pueda funcionar por los diferentes navegadores.
    \item Implementación de funcionalidades para que Grupos Intermedios puedan realizar Solicitudes de Fondos por Rendir.
    \item Lanzar el sistema como una herramienta oficial para las OE de los demás Campus que pertenecen a la Casa de Estudios.
    \item Realizar pruebas de seguridad.
\end{itemize}
