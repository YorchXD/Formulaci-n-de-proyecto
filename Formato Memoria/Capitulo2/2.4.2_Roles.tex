Para llevar a cabo el desarrollo del proyecto existen roles fundamentales dentro de la metodología Scrum, los cuales son:

\begin{itemize}
    \item   \begin{description}
                \item[Scrum Master (SM):] Encargado de que el equipo tenga todas las condiciones necesarias para trabajar sin obstáculos y así cumplir con el sprint. Por otra parte, es quien procura que el proceso siga su curso. Además, supervisa la comunicación dentro del equipo, resuelve desacuerdos dentro del equipo o entre los equipos, ayuda a asegurar la productividad del equipo. Cabe destacar que el scrum master no necesariamente es el líder para el equipo y es el responsable del ``cómo hacer''
            \end{description}    
    
    \item   \begin{description}
                \item[Product Owner (PO):] es el encargado de representar a los stakeholders (interesados externos o internos). La tarea más importante del PO es tomar decisiones sobre la realización y asignación de prioridad de los requisitos en la lista de tareas. Además, debe asignar requisitos del proyecto a los integrantes del equipo, ser extrovertido y tener facilidad para expresarse de forma oral y escrita. Por último, para que el PO pueda cumplir con sus funciones debe tener la capacidad de planificación y habilidades de comunicación, además de influir en el equipo con su interés y motivación.
            \end{description} 
    
    \item   \begin{description}
                \item[Team (equipo):] encargados de cumplir con el desarrollo del producto y están reunidos en equipos de cinco a diez personas. Tiene autoridad para reorganizarse y definir las acciones necesarias o sugerir remoción de impedimentos, los cuales deben ser discutidos y realizados al final del Sprint. Para ello, debe cumplir con ciertas características tales como la autogestión, la autoorganización y debe ser multifuncional.
            \end{description} 
    
    \item   \begin{description}
                \item[Stakeholders:] son todas aquellas personas beneficiadas y quienes posibilitan el proyecto. Estos pueden ser usuarios, administradores, técnicos, etc.
            \end{description}  
\end{itemize}
