Conceptos

Evento: 

Un evento según el "Plan estratégico de la Universidad de Talca Visión 2020" es una actividad emprendida por las distintas Organizaciones Estudiantiles reconocidas tales como recepción de alumnos nuevos, celebración del dia de la carrera, actividades culturales y deportivas, actividades sociales, entre otras ~\cite{5}.

Solicitud

Una solicitud es un documento en el cual se pide ayuda económica a quien dirige la Dirección de Escuela(en el caso de CAA) o a quien dirige la DAAE (en el caso de Federación) para llevar a cabo un evento organizado por la Organización Estudiantil. 

En el documento debe ir detallado:
 *el nombre del evento
 *la fecha de inicio y de término de la actividad 
 *el lugar a realizar la actividad 
 *el monto 
 *las categorías en que incurrirán los gastos
 *los datos de los participantes como el nombre y rut (en caso de que una actividad sea dirigida a un grupo de personas) 
 *los datos del encargado de la actividad (Presidente o Secretario de Finanzas de la Organización Estudiantil).

Resolución

Rendición
    Documentos(Boletas o facturas)

Tecnologías utilizadas

Lenguajes

HTML-Razor

HTML: Lenguaje de marcado de hipertextos que dispone de un conjunto de etiquetas que si son insertadas dentro de un documento este puede mostrar la información a través de un navegador web. [8]

CSS: Lenguaje que describe cómo el estilo de un documento HTML debe ser mostrado. [8]

Javascript: Lenguaje de programación interpretado, típicamente utilizado en el desarrollo web para implementar páginas web dinámicas que modifican el DOM de estas. [9]

C#

Frameworks
ASP.NET Core: es la nueva versión modular del framework .NET, la cual es multiplataforma y Open Source diseñado para la creación de aplicaciones modernas conectadas a Internet, como aplicaciones web y APIs Web. 
