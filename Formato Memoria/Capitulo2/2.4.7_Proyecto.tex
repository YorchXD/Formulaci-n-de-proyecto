En ítems anteriores se habla de la metodología que se debe aplicar al proyecto para lograr administrar y organizar los recursos. También se hablan de los roles que tienen los individuos que dan vida al proyecto pero, ¿Qué es el proyecto?

El proyecto es el ciclo general que ayuda a resolver el problema planteado. Como se aprecia en la \textbf{Figura \ref{fig: Proceso_Proyecto}}, comienza con una reunión de captura de requisitos y finaliza con la entrega del producto junto con la puesta en marcha. Dentro de este ciclo general contempla muchos sub-ciclos denominados ``Entregas'' o ``Incrementos'', que corresponden a nuevas funcionalidades realizadas y mejoras continuas a lo largo del proyecto. Para realizar estas entregas, existe otro sub-ciclo que corresponde a las ``Iteraciones'' o ``Sprints'' y dentro de esta, están enlazadas un conjunto de fases que se deben llevar a cabo para concluir cada iteración.

\begin{figure}[!htbp]
    \hspace{-9mm}
    \includegraphics[width=1.1\textwidth]{Imagenes/Proceso_Proyecto.png}
    \caption{\label{fig: Proceso_Proyecto}Proceso utilizado en el proyecto.}
\end{figure}