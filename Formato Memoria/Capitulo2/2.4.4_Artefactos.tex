\begin{itemize}
    \item   \begin{description}

                \item[Backlog: ] Lista de historias de usuarios que constituyen al proyecto. El PO evalúa estas historias y le da prioridad de esta lista de acuerdo con lo requerido.
                
                Existen tres tipos de backlogs, el product backlog, el backlog de versión o release backlog y por último el backlog de sprint, los que se muestran con mayor detalle a continuación:
                
                \begin{description}
                    \item[Product backlog: ] Lista de historias de usuarios o características del proyecto los cuales proporcionan valor comercial para los stakeholders. Estas historias de usuario se caracterizan por ser de alto nivel y no tienen detalles de implementación o técnicos.
                        
                    \item[Backlog de versión: ]  Consiste en una sub lista de tareas extraídas del backlog del producto, las cuales son priorizadas por el product owner y deben desarrollarse para la siguiente versión del producto. Estas deben ser más detalladas, con mayor precisión en cuanto al tiempo de desarrollo y puede tener observaciones realizadas por los miembros del equipo.
                        
                    \item[Backlog del sprint: ] Son tareas que provienen del backlog de versión y que pueden completarse en periodos cortos de tiempo. En cuanto al desarrollo de estas, deben completarse hasta el final del sprint por los equipos que se comprometieron en realizarlas.
                \end{description} 

                Aquellas historias de usuarios que no son completadas en su tiempo planificado, se devuelven al backlog donde son evaluadas nuevamente por el PO.
            \end{description}    

    \item   \begin{description}
                \item[Incremento:] Avance del proyecto que se realiza dentro del sprint en el cual los equipos se comprometieron a desarrollar.
            \end{description}
    
    \item   \begin{description}
                \item[Demostración:] Entrega de incremento del software a los stakeholders de tal manera que estos evalúen la funcionalidad implementada. Cabe destacar la posibilidad de que la demostración no contenga toda la funcionalidad planteada, sino que aquellas funciones susceptibles de entregarse dentro del periodo establecido.
            \end{description}
    
    \item   \begin{description}
                \item[Lista de impedimentos: ]  Son dificultades que impiden la productividad y la calidad del proyecto. El encargado de controlar esta lista, y en lo posible de que este vacía, es el scrum master, por lo cual crea una lista de tareas para tener seguimiento de los impedimentos que requieren resolverse. 
            \end{description}
\end{itemize}

