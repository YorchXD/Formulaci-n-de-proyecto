\begin{itemize}
    \item   \begin{description}
                \item[Backlog o retrasos: ] lista de historias de usuarios que se pueden agregar elementos a los retrasos en cualquier momento. El PO evalúa los retrasos y actualiza las prioridades de esta lista de acuerdo con lo requerido.
                
                Existen tres tipos, los cuales son el product backlog (retraso del producto), el backlog de versión o release backlog y por último el backlog de sprint (Retraso de sprint), los que se muestran con mayor detalle a continuación:
                
                \begin{description}
                    \item[Product backlog: ] lista de historias de usuarios o características del proyecto los cuales proporcionan valor comercial para los stakeholders. Estas historias de usuarios son de alto nivel y no pueden tener detalles de implementación o técnicos. Además, están asociados a una estimación de tiempo y son especificados según los criterios de la organización.
                     
                    \item[Backlog de versión: ]  consiste en una sub lista de tareas extraídas del backlog del producto, las cuales son priorizadas por el product owner y que deben desarrollarse para la siguiente versión del producto. Estas deben ser más detalladas, con mayor precisión en cuanto al tiempo en que demorará en desarrollarse y puede tener observaciones realizadas por los miembros del equipo.
                     
                    \item[Backlog del sprint: ] Son tareas que provienen del backlog de versión y que pueden completarse en periodos cortos de tiempo. En cuanto al desarrollo de estas, deben completarse hasta el final del sprint por los equipos que se comprometieron en realizarlas.
                \end{description} 
            \end{description}    

    \item   \begin{description}
                \item[Incremento:] avance del proyecto que se realiza dentro del sprint en el cual los equipos se comprometieron a desarrollar.
            \end{description}
    
    \item   \begin{description}
                \item[Demostración:] entrega de incremento del software a los stakeholders de tal manera que estos evalúen la funcionalidad implementada. Cabe destacar la posibilidad de que la demostración no contenga toda la funcionalidad planteada, sino que aquellas funciones susceptibles de entregarse dentro del periodo establecido.
            \end{description}
    
    \item   \begin{description}
                \item[Lista de impedimentos: ]  son dificultades que impiden la productividad y la calidad del proyecto. Para ello, el encargado de esto no suceda es el scrum master, por lo cual crea una lista de tareas para tener seguimiento de los impedimentos que requieren resolverse. 
        
            \end{description}
\end{itemize}

