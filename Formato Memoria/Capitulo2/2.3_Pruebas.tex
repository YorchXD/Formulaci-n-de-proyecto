Para el proyecto, se utiliza la metodología con experimentos controlados como método de evaluación. Esta maneja pruebas que se hacen bajo condiciones totalmente constante y en algunos casos, puede variar algún factor. El objetivo principal es realizar un muestreo de usuarios conocidos los cuales utilizan el software con pruebas de usabilidad simples, además del cliente.

Por otra parte, los usuarios que realicen las pruebas de usabilidad no necesariamente deben ser los mismos.

Para finalizar, el motivo por el cual se escoge esta metodología de evaluación se debe a que hace posible la comprensión e identificación de las variables que están siendo utilizadas en la construcción del software.  Todo lo anterior se debe ya que, según el artículo \emph{Albert Einstein and Empirical Software Engineering}~\cite{8}, experimentar con la construcción de software permite ``Aumentar la compresión de lo que hace al software bueno y cómo hacer un software bien''\footnote{Gain more understanding of what makes software ``good'' and how to make software well.}.