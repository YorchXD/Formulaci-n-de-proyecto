Para cada iteración se cuenta con las siguientes fases:
\begin{enumerate*}[label=(\roman*)]
    \item Requisito,
    \item Diseño,
    \item Planificación,
    \item Construcción y
    \item Pruebas.
\end{enumerate*}
Sin embargo, su presencia y tiempo de dedicación a cada una depende de cada iteración. Por ejemplo, en la primera iteración se incluyen todas las fases debido a que se realiza un prototipo del proyecto. Mientras que existen otras en que solo serán para corregir o modificar alguna funcionalidad. Todo lo anterior depende de las condiciones presentes en la iteración.

A continuación se procede a contextualizar de qué trata cada una de las fases de la iteración:

\paragraph{Fase de requisitos: } en esta etapa es donde el PO se reune con el cliente para obtener información sobre las funcionalidades que tiene el proyecto (etapa comúnmente denominada ``Captura de Requisitos’’). En la metodología Scrum, para este efecto existe lo que se conoce como \emph{historia de usuario}, en donde, de manera simple y clara se especifica semiformalmente lo que el cliente espera de la aplicación. A mayor abundamiento, las historias de usuario son pequeñas descripciones de los requerimientos de un cliente; y para su redacción se debe considerar la descripción del rol, la funcionalidad y el resultado esperado en una frase cortas, cumpliendo con el siguiente formato: \textbf{Como $<$quién$>$ Quiero $<$qué$>$ Para $<$objetivo$>$}. Naturalmente, para una aplicación se pueden definir decenas o miles de historias de usuario, dependiendo de la envergadura del proyecto. Para esta fase se recomienda hacer pautas para mejorar el trabajo con el cliente. Entonces, lo que debe realizar el PO es escribir lo que el cliente quiere como \textbf{\emph{historias de usuario}}.


\paragraph{Fase de diseño: } tras la obtención de requisitos, se procede a realizar el diseño del sistema, el cual incluye la base de datos, la interfaz y la estructura. Cabe destacar que cada diseño debe ser simple de comprender.

\paragraph{Fase de planificación: } Luego de tener claro de lo que se quiere al realizar la captura de requisitos y el diseño, se procede a describir las actividades a realizar. Para cada actividad se debe definir su duración, dependencia y la fecha de inicio y de fin. Por último, se debe asignar estas tareas a un responsable de algún equipo de trabajo. 

\paragraph{Fase de construcción: } ya asignadas las tareas a los responsables, se procede a codificar e implementar las funcionalidades que logran cumplir con los requisitos planteados en la planificación. La finalidad de esta etapa es desarrollar un incremento como se ha explicado en la \textbf{Sección \ref{sec:Artefactos}} a través de iteraciones sucesivas.

\paragraph{Fase de pruebas: } En esta fase se evalúa la funcionalidad del incremento. Para ello se hacen pruebas de integración y validación junto con el cliente, quien realiza las pruebas de aceptación.
