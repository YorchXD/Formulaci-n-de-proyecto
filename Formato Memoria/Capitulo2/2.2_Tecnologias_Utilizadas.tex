Para resolver el problema mencionado en el Capítulo \ref{sec:Intro}, existen muchas tecnologías dentro de las cuales se encuentra, PHP, Ruby on rails, WordPress, Wix, entre otras. Pero para este proyecto se utilizó ASP.NET Core cuyo motivo se debe a que es un lenguaje sólido y robusto, tanto para proyectos pequeños, grandes o con gran potencial de crecimiento. Además, está orientado a objeto debido a que C\# conforma parte de la tecnología, haciendo que el código sea ordenado y legible. Por otra parte, se puede utilizar para programar aplicaciones móviles, web o incluso para aplicaciones de computadoras de escritorio. Por último, otra de las razones de escoger esta tecnología es porque las empresas del rubro la están solicitando como conocimientos requeridos. 

Para entrar un poco más en detalle, se dará a conocer pequeñas descripciones de los principales componentes que cuenta esta tecnología, además de las librerías que se utilizó en el proyecto.

\begin{itemize}
    \item   \begin{description}
                \item[HTML-Razor:] HTML es un lenguaje de marcado de hipertextos el cual dispone de un conjunto de etiquetas las cuales son insertadas en un documento que puede mostrar la información a través de un navegador web. Este cuenta ademas con Razor, el cual es una sintaxis basada en C\# que permite usarse como motor de programación en las vistas html. 
            \end{description}

    \item   \begin{description}
                \item[CSS:] Lenguaje de diseño gráfico que describe cómo será la presentación de un documento estructurado escrito en HTML. 
            \end{description}

    \item   \begin{description}
                \item[Javascript:] Lenguaje de programación interpretado, el cual es utilizado en el desarrollo web para implementar páginas web dinámicas.
            \end{description}

    \item   \begin{description}
                \item[C\#:] Lenguaje de programación orientado a objetos el cual fue desarrollado por Microsoft para la plataforma dotNET. Su sintaxis básica proviene de c/c++ y es similar a java e incluye mejoras cuyo origen son de otros lenguajes.
            \end{description}
    
\end{itemize}

