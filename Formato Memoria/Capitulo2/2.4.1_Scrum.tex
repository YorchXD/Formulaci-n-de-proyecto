Scrum\footnote{Nombre originario  de una jugada que tiene lugar durante un partido de rugby en donde se forma un grupo de jugadores alrededor del balón y todos trabajan juntos (a veces con violencia) para moverlo a través del campo. } es un método de desarrollo ágil de software creado por Jeff Sutherland y su equipo de trabajo al inicio de la década de los 90~\cite{7}.

Su objetivo principal es ser un proceso iterativo e incremental que ayuda a la gestión de proyectos con requisitos cambiantes. Dentro de sus principales cualidades se encuentran las iteraciones y reuniones en el proceso de desarrollo con actividades que ayudan a controlar los cambios y los riesgos del proyecto para elevar la productividad.

Scrum produce resultados en periodos cortos de tiempos (aproximadamente 30 días), delegando al equipo de trabajo la responsabilidad de escoger la mejor forma de trabajar.

La clave del éxito de Scrum son los \emph{sprints}, los cuales pueden durar de dos a cuatro semanas, en donde al iniciar cada uno de estos, se seleccionan una lista de historias de usuarios (backlog). Por otra parte, dentro del transcurso de cada iteración, diariamente se realizan breves reuniones de aproximadamente quince minutos, las que ayudan a conocer el avance del desarrollo del proceso.

%Por otra parte, es muy comun que en Scrum hable de \textbf{\emph{historias de usuario}}, ya que es de mucha importancia. Estas son pequeñas descripciones de los requerimientos de un cliente. Para su redacción se debe considerar la descripción del rol, la funcionalidad y el resultado esperado en una frase cortas, cumpliendo con el siguiente formato: \textbf{Como $<$quién$>$ Quiero $<$qué$>$ Para $<$objetivo$>$}.