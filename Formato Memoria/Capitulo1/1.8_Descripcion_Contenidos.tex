El resto de este documento tiene la siguiente organización:

\begin{itemize}
    \item El capítulo dos hace referencia a los antecedentes necesarios para la comprensión del contexto en que se desarrolla el proyecto.
    \item En cuanto al capítulo tres, se hace mención de la metodología que se utiliza, además de la forma en que se organizan las iteraciones y el orden en que se desarrollan estas para tener una aproximación del tiempo que conlleva el desarrollo del proyecto.
    \item En el capítulo cuatro se detalla el modelo de contexto el cual ayuda a la compresión de la situación actual del sistema y cuál es su enfoque. Tras lo anterior se procede a realizar encuestas para obtener las características del sistema y realizar la planificación general del proyecto.
    \item El capítulo cinco se enfoca en el diseño de la aplicación. Comienza con el Mapa de Navegación que tiene la aplicación junto con el diseño general de la interfaz. Luego se detalla la arquitectura del sistema, para así diseñar el modelo de datos, modelo relacional de la aplicación y por último el algoritmo de solución para la selección automática de documentación que requiere la rendición.
    \item El capítulo seis trata sobre la aplicación obtenida tras las iteraciones planificadas en el capítulo cuatro. Además, se menciona las pruebas realizadas y los resultados obtenidos tras el uso de la aplicación por los usuarios.
    \item Por último, el capítulo siete menciona las conclusiones del proyecto en base al trabajo realizado y los resultados obtenidos. Además, se da a conocer los trabajos a futuro a desarrollar.
    \item Junto con esto, se incluyen dos anexos que contienen el detalle de las pruebas de las funcionalidades en modalidad de caja negra y los resultados de las pruebas de usabilidad.
\end{itemize}
