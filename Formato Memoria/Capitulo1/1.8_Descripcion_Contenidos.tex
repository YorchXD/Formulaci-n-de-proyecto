El resto de este documento tiene la siguiente organización:

\begin{itemize}
    \item El capítulo dos hace referencia a los antecedentes necesarios para la comprensión del contexto en que se desarrolla el proyecto.
    \item En cuanto al capítulo tres, se hace mención de la metodología que se utiliza, además de la forma en que se priorizan estos últimos para así saber una aproximación del tiempo que conlleva el desarrollo del proyecto.
    \item El capítulo cuatro se detallan el modelo de contexto el cual ayuda a la compresión de la situación actual del sistema y en que se enfoca el sistema. Tras lo anterior se procede a realizar encuestas para obtener las características del sistema y realizar la planificación general del proyecto.
    \item El capítulo cinco, se enfoca en el diseño de la aplicación en el cual comienza con el Mapa de navegación que tiene la aplicación junto con el diseño general de interfaz. Luego se detalla la arquitectura del sistema, para asi diseñar el modelo de datos y por último el modelo relacional de la aplicación.
    \item El capítulo seis trata sobre la aplicación obtenida tras las iteraciones planificadas y que se mencionan el capítulo cuatro. Además, se menciona las pruebas realizadas y los resultados obtenidos tras el uso de la aplicación por los usuarios.
    \item El capítulo siete se mencionan las conclusiones del proyecto en base al trabajo realizado y los resultados obtenidos. Además, se da a conocer los trabajos a futuros a desarrollar.
\end{itemize}