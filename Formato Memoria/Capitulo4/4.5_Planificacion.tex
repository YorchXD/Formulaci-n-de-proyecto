Tras realizar el estudio previo de la situación actual de la problemática y conocidas las historias de usuario, se procede a implementar la segmentación del proyecto enunciada en la \textbf{Sección \ref{sec:Iteracion}}, para luego planificar el desarrollo de la aplicación. Es por ello que el \textbf{Cuadro \ref{tab: tab_planificacion}} muestra cómo se distribuyen las historias de usuarios a través de las diferentes iteraciones.


\begin{table}[htbp]
    
    \caption{\label{tab: tab_planificacion} Planificación de las iteraciones. }
	\footnotesize
	\centering
    \begin{tabular}{|l|l|}
    
    	\hline
    	\textbf{Iteración} & \textbf{Historia de usuario} \\
    	\hline\hline
    
		1. Solicitud	& 	HU-PDTE-10\\ \cline{2-2}
					&	HU-SF-5 \\ \cline{2-2}
					&	HU-SF-4 \\ \hline

		2. Solicitud y Resolución	& 	HU-PDTE-9\\ \cline{2-2}
					&	HU-SF-10 \\ \hline
					
		3. Búsqueda	& 	HU-PDTE-4 \\ \hline

		4. Rendición	& 	HU-PDTE-3\\ \cline{2-2}
					&	HU-PDTE-2 \\ \cline{2-2}
					&	HU-SF-7 \\ \cline{2-2}
					&	HU-SF-6 \\ \cline{2-2}
					&	HU-SF-8 \\ \cline{2-2}
					&	HU-SF-3 \\ \cline{2-2}
					&	HU-SF-2 \\ \cline{2-2}
					&	HU-PDTE-5 \\ \cline{2-2}
					&	HU-SF-9 \\ \hline

		5. Usuario	& 	HU-PDTE-11 \\ \hline
		
		6. Documentación y Panel de Ayuda	& 	HU-PDTE-6\\ \cline{2-2}
					&	HU-PDTE-8 \\ \hline

		7. Detalles	& 	HU-PDTE-1\\ \cline{2-2}
					&	HU-SF-1 \\ \cline{2-2}
					&	HU-PDTE-7 \\ \hline
    \end{tabular}  
\end{table}