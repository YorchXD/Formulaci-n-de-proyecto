Después conocer los \textbf{Aspectos generales} y los \textbf{Modelos de contexto de solicitudes}, se procede a presentar los requerimientos del sistema como ``Historias de Usuario'' las cuales se pueden observar en los \textbf{Cuadros \ref{tabla:Historias_Ususario_Secretario_Finanza}} y \textbf{\ref{tabla:Historias_Ususario_Presidente}}. Estas fueron obtenidas a través de entrevistas a Presidentes y Secretarios de Finanzas de las respectivas OE, debido a que son los únicos que pueden realizar los procesos de solicitud de fondos por rendir. Además fueron confirmadas por las respectivas Secretarias de Escuelas, ya que ayudan a detectar alguna anomalía en las solicitudes y/o rendiciones antes de ser enviadas a los funcionarios encargados de decidir su aprobación. Cabe destacar que las historias de usuario que se muestra en los \textbf{Cuadros \ref{tabla:Historias_Ususario_Secretario_Finanza}} y \textbf{\ref{tabla:Historias_Ususario_Presidente}} se redactan según el primero que la menciona.

\begin{table}[htbp]
    \centering
    \caption{Historias de usuario - Secretario de Finanza}
    \label{tabla:Historias_Ususario_Secretario_Finanza}
    \begin{tabular}{| p{2.6cm}| p{12.2cm} |}
    \hline
    Código & Yo, como Secretario de Finanza quiero... \\
    \hline \hline
    
    HU-SF-1 & que sólo se pueda editar una rendición sin llegar a su estado de finalizado dentro del periodo de gracia (20 días contados desde que finaliza el evento), para que no sea rechazada por contraloría por exceso de tiempo. \\ \hline

    HU-SF-2 & ingresar los gastos incurridos de un evento (montos de boletas y/o facturas) para que se calcule el óptimo más cercano al monto solicitado. \\ \hline

    HU-SF-3 & que para una boleta y/o factura se ingrese la fecha del documento, nombre del proveedor, breve descripción del gasto y monto del gasto para cumplir con lo solicitado con la resolución. \\ \hline

    HU-SF-4 & ingresar a los participantes de un evento, para detallar los gatos que estos realizaron en el evento (si es el caso). \\ \hline

    HU-SF-5 & ingresar el rango de tiempo en que ocurre el evento, para no ingresar boletas/facturas que no estén dentro de la fecha. \\ \hline

    HU-SF-6 & que una vez validada la rendición pueda ser exportada a PDF, para poder ser revisada por el/la Director/a de Escuela o Decano según la OE que corresponda, para poder ser enviada a contraloría. \\ \hline

    HU-SF-7 & tener un listado de los documentos que debo adjuntar la documentación requerida en la rendición en el orden en que se encuentra en esta, para poder chequear que se están adjuntando en el mismo orden. \\ \hline

    HU-SF-8 & poder editar una rendición en caso de ser rechazada por errores, para poder solucionarlos. \\ \hline

    HU-SF-9 & que las boletas este en categorías, para saber en si los gastos son de alojamiento, alimentación, gasto de incorporación e inscripción, etc. \\ \hline

    HU-SF-10 & guardar los datos de la RU en la cual se aprueba la solicitud de fondos por rendir realizado por la OE además de guardar una copia digitalizada de esta, para tener un respaldo.\\ \hline
    \end{tabular}
\end{table}

\begin{table}[htbp]
    \centering
    \caption{Historias de usuario - Presidente}
    \label{tabla:Historias_Ususario_Presidente}
    \begin{tabular}{| p{2.6cm}| p{12.2cm} |}
    \hline
    Código & Yo, como Presidente quiero... \\
    \hline \hline
    HU-PDTE-1 & tener respaldo de rendición anteriores finalizadas, para poder visualizarlas en cualquier instante. \\ \hline

    HU-PDTE-2 & visualizar y editar rendiciones sin terminar, para ver que cosas faltan y si es posible terminarla. \\ \hline
   
    HU-PDTE-3 & visualizar la cantidad de días disponibles para realizar la rendición. \\ \hline

    HU-PDTE-4 & realizar búsquedas de solicitudes por fecha o por el nombre del evento para poder encontrarlas y verlas en detalle. \\ \hline

    HU-PDTE-5 & que se valide que las boletas y/o facturas ingresadas estén dentro del periodo del evento, para no rechacen la rendición. 
    \\ \hline

    HU-PDTE-6 & adjuntar copias digitales en formato PDF de los documentos solicitados en la rendición (boletas/facturas), para tener un respaldo de lo que se enviará a revisión. \\ \hline

    HU-PDTE-7 & visualizar el estado de un fondo por rendir, para saber si es factible ejecutar la rendición del fondo. \\ \hline

    HU-PDTE-8 & visualizar la forma de como realizar una rendición, para saber como realizarlas. \\ \hline

    HU-PDTE-9 & realizar el documento de solicitud de fondos por rendir del evento, para tener un registro de este. \\ \hline

    HU-PDTE-10 & ingresar en la solicitud el monto del evento, participantes y categorías (alojamiento, alimentación, etc) en al que se invertirá el fondo, encargado del evento y a quien irá dirigida la solicitud, para generar el documento de solicitud de fondos. \\ \hline

    HU-PDTE-11 & que existan cuentas para federación y centros de alumnos, para poder realizar y visualizar rendiciones. \\ \hline

    \end{tabular}
\end{table}





    

    