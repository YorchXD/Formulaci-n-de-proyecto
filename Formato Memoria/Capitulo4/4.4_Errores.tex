%Si bien se conoce el problema principal explicado en capitulos anteriores junto con el contexto en que se desenvuelve y además conociendo las historias de usuario expuestos en los \textbf{Cuadros \ref{tabla:Historias_Ususario_Presidente}} y \textbf{\ref{tabla:Historias_Ususario_Secretario_Finanza}}, se han detectado los siguientes problemas:


Luego de conocer el contexto en el que se desenvuelve el problema expuesto en la \textbf{Sección \ref{sec: Contexto_Solicitud}} junto con las historias de usuario expuestas en los \textbf{Cuadros \ref{tabla:Historias_Ususario_Secretario_Finanza}} y \textbf{\ref{tabla:Historias_Ususario_Presidente}}, se realiza una reflexión acerca de los tipos de errores que existen en el proceso manual de Gestión de Fondos.

A partir de eso, se ha detectado varias fuentes de inconvenientes en el proceso, las que se listan a continuación: \begin{enumerate*}[label=(\roman*)]
    \item Rechazo de la declaración de gastos,
    \item El monto total de los gastos declarados no es óptimo (en el sentido de que queda dinero disponible que no se utiliza),
    \item Asignación incorrecta de gastos,
    \item Declaración de gastos realizada fuera de plazo,
    \item Documentación adjuntada sin chequear,
    \item Gastos categorizados incorrectamente,
    \item Falta de respaldo de documentación,
    \item Formato incorrecto de la declaración de gastos y
    \item Desconocimiento del tipo de declaración de gasto (Solicitud de reembolso/Rendición de gastos).
\end{enumerate*}

Por último, el detalle exhaustivo de errores que se considera prevenir con el sistema desarrollado en este trabajo se muestra en el \textbf{Cuadro \ref{tab: Errores_frecuentes_Fondo}}, con su respectivo código.


\begin{table}[htbp]
    \centering
    \caption{Errores frecuentes del proceso de Solicitud de Fondos}
    \label{tab: Errores_frecuentes_Fondo}
    \begin{tabular}{| p{2.6cm}| p{12.2cm} |}
    \hline
    Código & Problema \\
    \hline \hline
    
    PBLM-1 & Rendición o Solicitud de Reembolso rechazada por realizarla fuera de plazo de 20 días corridos. \\ \hline

    PBLM-2 & Suma de boletas y/o facturas no optimizada. \\ \hline

    PBLM-3 & Asignar un gasto a un participante que no esté en la solicitud. \\ \hline

    PBLM-4 & Documentación realizada fuera del periodo en que ocurre el evento. \\ \hline

    PBLM-5 & Falta de chequeo de los documentos que se deben adjuntar en la Rendición/Solicitud de reembolso. \\ \hline

    PBLM-6 & Rendir gastos fuera de las categorías expuestas en la Solicitud. \\ \hline

    PBLM-7 & Ausencia de respaldo de la RU y del documento de declaración de gastos de los Fondos Solicitados. \\ \hline

    PBLM-8 & No respetar el formato de Rendiciones/Solicitud de reembolso. \\ \hline

    PBLM-9 & No indicar el tipo de declaración de gastos, por ejemplo si es una Solicitud de reembolso o es una Rendición de gastos. \\ \hline
    \end{tabular}
\end{table}